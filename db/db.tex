\documentclass[a4paper]{article}
\usepackage[utf8]{inputenc}
\usepackage[latin1]{inputenc}
\usepackage[russian]{babel}
\usepackage{graphicx}
\usepackage{url}
\usepackage{amsmath}
\usepackage{amssymb}
\graphicspath{ {pictures/} }
\begin{document}
\begin{center}
        Лекции по базам данным.
\end{center}
\section*{Тема №1: Основные понятия и определения}
Рассматриваемые вопросы:
\begin{itemize}
        \item Предметная область
	\item Данные, информация, знания
	\item Модель представляемых данных
	\item Базы данных и системы управления базами данных
	\item Системы баз данных
	\item Словарь базы данных
\end{itemize}
\textbf{Предметная область} - это часть реального мира, подлежащая изучению для достижения поставленной цени (автоматизации, управления, создания базы данных и др.)\\
Предметная область представляется множеством фрагментов. Каждый фрагмент предметной области. Каждый фрагмент предметной области характеризуется множеством объектов и процессов, использующих объекты, а также множеством пользователей, характеризуемых различными взглядами на предметную область.\\
Например, предметная область ВМФ - базы, корабли, заводы... Корабль - тип, порт приписки, траектории движения.\\\\
\textbf{Модель предметной области} - концептуальная (семантическая) модель, предназначенная для представления семантики предметной области на самом высоком уровне абстракции.\\
Это означает, что устранена или минимизирована необходимость использовать понятия "низкого уровня" связанные со спецификой физического представления и хранения данных.\\
Наиболее известным представителем класса семантических моделей является модель "сущность-связь" (ER-модель).\\\\
\textbf{Данные} - это набор конкретных значений параметров или факторов, характеризующих объект, условие, ситуацию и т.д.\\
\textbf{Информация} - совокупность данных, упорядоченная с определённой целью, придающей ей смысл.\\
\textbf{Знания} - это закономерности (принципы, связи, законы) предметной области, позволяющие специалистам ставить и решать задачи в этой области.\\ 
\textbf{Модель представления данных} - набор принципов, которые определяют:
\begin{enumerate}
        \item организацию логической структуры хранения данных
	\item методы манипулирования данными
	\item методы поддержки целостности данных
\end{enumerate}
Примеры моделей данных:
\begin{itemize}
        \item иерархическая модель данных
	\item сетевая модель данных
	\item реляционная модель данных
	\item многомерная модель данных
	\item объектная модель данных
	\item объектно-реляционная модель данных
	\item модель инвертированных списков
	\item графовая модель
	\item и др.
\end{itemize}
\textbf{База данных} - логически структурированная совокупность постоянно хранимых данных, характеризующих актуальное состояние предметной области и используемых прикладными программными системами.\\
Включает также метаданные (словарь данных/системный каталог) и другую служебную информацию.\\
\textbf{База данных} - совокупность программно-технических средств, предназначенных для хранения данных и предоставления пользователю необходимой ему информации.\\
\textbf{Система управления базами данных} - это совокупность программных средств, предназначенная для модификации и извлечения из базы данных необходимой пользователю информации, а также для создания баз данных, поддержания их в работоспособном состоянии, обеспечения безопасности баз данных и решения других задач администрирования.\\
\begin{center}
	\includegraphics[scale=0.3]{1}
\end{center}
\textbf{Информационная система} - система, которая предназначеная для хранения, поиска и обработки информации, соответствующие информационные ресурсы (человеческие, технические, финансовые и т.д.), которые обеспечивают и распространяют информацию, в целях поддержки определённой деятельности в рамках конкретной предметной области.\\\\
Понятие \textbf{система баз данных} (банк данных) используется как в широком, так и в узком смысле.\\
В широком смысле система баз данных понимается фактически как синомим понятия информационная система и включает в себя данные, аппаратное обеспечение, программное обеспечение и пользователей.\\
В узком смысле система баз данных понимается как СУБД с управляемой ей базой данных.
\begin{center}
	\includegraphics[scale=0.3]{2}
\end{center}
\textbf{Словарь баз данных} представляет собой набор системных таблиц и представлений.\\
Таблицы словаря данных хранят метаданные базы данных - информацию обо всех объектах базы данных (имена таблиц, типы данных столбцов, имена пользователей, сведения о процедурах, функциях и т.д.)\\
Создание или модификация любого объекта данных вызывает обновление таблиц словаря базы данных, который отражает сделанные изменения.\\
Пользователь получает доступ к таблицам словаря данных при помощи предопределенных представлений вместо того, чтобы читать таблицы непосредственно.\\
Виды пользователей баз данных.
\textbf{Конечные пользователи}:
\begin{itemize}
        \item Нуждаются в наличии удобных, простых и эффективных средств работы с базой данных, не требующих от пользователей образования в области информационных технологий или длительного этапа специальных подготовки и обучения
	\item Могут использовать язык запросов к БД.
\end{itemize}
\textbf{Прикладные программисты}:
\begin{itemize}
	\item Занимаются написанием, отладкой и внедрением прикладных программ (приложений), использующих информацию из базы данных
	\item Используют язык запросов к БД для получения и изменения данных и различные языки программирования для реализации процедур обработки данных
\end{itemize}
\textbf{Администраторы данных}:
\begin{itemize}
	\item Определяет требования по содержанию и защите данных
	\item Не обязаны быть специалистом в области ИТ (аналитики и руководители
\end{itemize}
\textbf{Администраторы баз данных}:
\begin{itemize}
        \item Реализует сложные задачи проектирования, создания, организации и поддержки работы систем баз данных в соответствии с решениями администраторов данных
	\item Являются специалистами в области информационных технологий
\end{itemize}
\textbf{Функции администратора БД}:
\begin{itemize}
        \item Анализ предметной области, для которой создаётся база данных
	\item Проектирования логической структуры БД
	\item Определение правил поддержания данных в согласованном состоянии (ограничений целостности)
	\item Первоначальная загрузка и ведение БД
	\item Реализация механизмов защиты данных
	\item Резервное копирование и восстановление БД после сбоей
	\item Анализ функционирование БД с возможной её модернизацией с целью увеличения производительности
	\item Взаимодействие с конечными пользователями и прикладными программистами
\end{itemize}

\end{document}
