\documentclass[a4paper]{article}
\usepackage[utf8]{inputenc}
\usepackage[russian]{babel}
\begin{document}
\begin{center}
	Лекция по вычислительной математике №1.
\end{center}
\section*{Список литературы:}
Учебное пособие. Пак. В. Г., Черкассова\\
Демидович Б. П. - Основы вычислительной математики\\
Амосов. А. А. - Вычислительные методы для инженеров\\
Мысовских И. П. - Лекции по методам вычислений\\
Фаддеев Д. К. - Вычислительные методы линейной алгебры\\\\
Задачники:\\
Бахвалов Н. С. - Численные методы в задачах и упражнениях\\
Копчёнова Н. В. - Вычислительная математика в примерах и задачах\\
\newpage

\section*{Введение}

Вычислительная математика - прикладной раздел математики, в котором разрабатываются и исследуются методы численного решения типовых математических задач с применением компьютеров.\\
\begin{center}
Особенности современных прикладных вычислительных задач:\\
\end {center}
1. Сложность математических моделей, реальных объектов, систем, процессов\\
2. Значительно возросший объём обрабатываемых данных и необходимых вычислений\\
3. Сложность применяемых методов вычислений и широкое использование алгоритмов модульной структуры\\
\begin{center}
	Этапы численного решения прикладной вычислительной задачи:
\end{center}
1. Постановка задачи (Формулировка на языке предметной области этой задачи)\\
2. Математическое моделирование задачи (Перевод на математический язык)\\
3. Постановка вычислительной задачи - конкретизация и выделение вычислительных свойств, точная формулировка требуемого результата\\
4. Выбор (создание) численного метода решения\\
5. Алгоритмизация и программирование\\
6. Получение и анализ результата\\
7. Коррекция математической модели или исходной задачи (При неудовлетворительном результате)\\ 
\newpage
\section*{Элементарная теория погрешностей.}
\setcounter{section}{1}
Все величины с которыми приходится иметь дело в вычислительной математике - являются приближёнными.\\
\subsection{Источники и классификация погрешностей численных значений}%
\setcounter{subsection}{2}
Источники погрешностей:\\
1. Приближённость математических моделей\\
2. Погрешности исходных данных\\
3. Приближённость методов численного решения\\\\
4. Неизбежные потери точности при машинном представлении чисел и арифметических операциях над ними\\
Погрешности результатов делятся на устранимые (причины 1 и 2) и неустранимые (причины 3 и 4). 
\subsection{Абсолютные и относительные погрешности}%
\setcounter{subsubsection}{3}
Пусть $a$ - точное, $a^*$ - приближённое значение некоторой скалярной величины.\\\\
Определение. Погрешностью приближённого значения $a^*$ называется:\\
$\varepsilon a^* = a - a^*$\\\\
Определение. Абсолютной погрешностью $a^*$ называется:\\
$\Delta a^* = |\varepsilon a^*| = |a - a^*|$\\\\
Верхняя оценка абсолютной погрешности обозначается $\bar{\varepsilon}a^*$, то есть $\bar{\varepsilon}a^*$ - такое число, про которое заведомо известно, что $\varepsilon a \leq \bar{\varepsilon} a*$\\\\
Определение. Относительной погрешностью приближённого значения $a^*$ называется отношение величин:\\
$\delta a^* = \frac{}{}$\\

\subsection{Значащие цифры в десятичной записи числа}%
\setcounter{subsection}{4}
Приближённые числа записываются в виде конечныых десятичных дробей, возможно с порядком, т.е. в виде:\\
$\frac{a_n a_{n-1} \dots a_0 \beta }{m}$\\
Значащими цифрами считаются все цифры мантиссы, начинающиеся с первой ненулевой слева.\\
Замечание - нули справа убирать нельзы, поскольку они означают разряды числа, например, $0,56$ и $0,560$ - разные приближённые числа, так как первое дано с двумя знаками после запятой, а второе - с тремя.\\\\\\
Есть два правила округления:\\
По усечению - просто отсечение всех цифр, до которой мы округляем\\
По дополнению - \\
\end{document}
